\PassOptionsToPackage{table}{xcolor}
\documentclass[usenames,dvipsnames, table]{beamer}
\usepackage{verbatim}
\usepackage[utf8]{inputenc}
\usepackage{graphicx}
\usepackage{./theme/beamerthemepresentation}
\addtobeamertemplate{footnote}{}{\vspace{2ex}}
\usepackage{amsmath}
\usepackage{stmaryrd}
\usepackage{amssymb}
\usepackage{mathtools}
\usepackage{mathtools}
\usepackage{tikz}
\usepackage{mathptmx}
\usepackage{changepage}
\usepackage{listings}
\usepackage{textcomp}
\usepackage{subcaption}
\usepackage{semantic}
\usepackage{marvosym}
\usepackage{bm}

\usetikzlibrary{calc, arrows.meta, intersections, patterns, positioning, shapes.misc, fadings, through,decorations.pathreplacing, math}
\usetikzlibrary{arrows,automata,positioning, calc, arrows.meta, chains, decorations.pathreplacing, shapes.geometric, shapes.arrows, shadows.blur}
%%% Bibliography
\usepackage[style=authoryear,backend=biber, style=alphabetic]{biblatex}
\addbibresource{references.bib}

\DeclarePairedDelimiter\Brackets{\lBrack}{\rBrack}

\mode<presentation> { \setbeamercovered{transparent} }
\setbeamertemplate{navigation symbols}{}
\def\beamerorig@set@color{%
  \pdfliteral{\current@color}%
  \aftergroup\reset@color
}
\def\beamerorig@reset@color{\pdfliteral{\current@color}}
\newcounter{bibitmctr}
\newcommand{\brf}{%
  \stepcounter{bibitmctr}%
  \ifnum\value{bibitmctr}=5%
    \setcounter{bibitmctr}{0}
    \framebreak
  \fi
}

\usepackage{minted}
\usepackage{xspace}
\usepackage{ragged2e}
\usepackage{pgfplots}
\usepackage{soul}
\usepackage{pgfplotstable}
\pgfplotsset{compat=1.16}
\usepgfplotslibrary{groupplots}
\usepackage{booktabs}
\usepackage{array}
%\usepackage{beramono}
%\usepackage[scaled=0.8]{FiraMono}

%%% Some useful commands
% pdf-friendly newline in links
\newcommand{\pdfnewline}{\texorpdfstring{\newline}{ }}
% Fill the vertical space in a slide (to put text at the bottom)
\newcommand{\framefill}{\vskip0pt plus 1filll}

\renewcommand*{\bibfont}{\scriptsize}

\newenvironment<>{varblock}[2][\textwidth]{%
  \setlength{\textwidth}{#1}
  \begin{actionenv}#3%
    \def\insertblocktitle{#2}%
    \par%
    \usebeamertemplate{block begin}}
  {\par%
    \usebeamertemplate{block end}%
  \end{actionenv}}

\TPGrid[0mm,0mm]{20}{20}
\definecolor{codebg}{HTML}{eeeeee}
\setminted{fontsize=\small, escapeinside=!!, bgcolor=codebg}
\usemintedstyle{tango}

%%%%%%%%%%%%%%%%%%%%%%%%%%%%%%%%%%%%%%%%%%%%%%%%%%%%%%%%%%%%%%%%%%%%%%%%%%%%%%%%
%%%%%%%%%%%%%%%%%%%%%%%%%%%%%%%%%%%%%%%%%%%%%%%%%%%%%%%%%%%%%%%%%%%%%%%%%%%%%%%%
%%%%%%%%%%%%%%%%%%%%%%%%%%%%%%%%%%%%%%%%%%%%%%%%%%%%%%%%%%%%%%%%%%%%%%%%%%%%%%%%
%%%%%%%%%%%%%%%%%%%%%%%%%%%%%%%%%%%%%%%%%%%%%%%%%%%%%%%%%%%%%%%%%%%%%%%%%%%%%%%%

\setcounter{secnumdepth}{2}
\setcounter{tocdepth}{2}

\title[Modular I/O Reasoning in DimSum]{Modular I/O Reasoning in DimSum}
\date[March 2025]{March, 2025}
\author[]{
  {\large Alex Loitzl${}^{1}$}
  \pdfnewline
  \pdfnewline
  \pdfnewline
  \pdfnewline
  \pdfnewline
  \pdfnewline
  \pdfnewline
  {\small ${}^{1}$Institute of Science and Technology Austria (ISTA)}
	%180.773 Seminar for Master Students in L\&C
}

\begin{document}
\addtocounter{framenumber}{0}

\begin{frame}
\titlepage
\end{frame}
% Title Slide
% Basic CompCert slide
% Example
% Challgenges
% Register Files
% Design slide
% Tables/Evaluation
% List of contributions

\section{Background}

\begin{frame}{Modular I/O Reasoning}
\end{frame}

\begin{frame}{Multi-language Reasoning in DimSum}
\end{frame}

\section{Project}
\begin{frame}{Rotation Project}
\end{frame}

\begin{frame}{Summary}
  \begin{itemize}
    \item Formally verified compiler
          \begin{itemize}
            \item Proof covers all optimizations
            \item Correct w.r.t. the modeled semantics
          \end{itemize}
    \item Discrepancies between hardware and model
          \begin{itemize}
            \item Cannot implement correct calling conventions
            \item Cannot support TriCore architecture
          \end{itemize}
    \item Suboptimal code generation
          \begin{itemize}
            \item Inserted moves
            \item Higher register pressure
          \end{itemize}
  \end{itemize}
\end{frame}

\begin{frame}
  \begin{textblock}{20}[0.5,0.5](10,12)
    \begin{figure}
    \centering
    \begin{tikzpicture}
  \useasboundingbox (-1.6,-3) rectangle (11.1,4);

  \node<1>[] (l) {\Large$\llbracket$\textcolor{red}{$\text{echo}_{\text{rec}}$}};
  \node (plus) at (1.63,0) {\Large$\bm{\oplus}$};
  \node<1> (r) at (3.9,0) {\Large$\text{getc}_{\text{spec}} \rrbracket$};
  \node (sim) at (6.4,0) {\Large$\boldsymbol{\preceq}$};
  \node<1> (spec) at (8.8, 0) {\Large$\llbracket \text{echo}_{\text{spec}} \rrbracket$};

  %%%%%%%%%%%%%%%%%%%%%%%%%%%%%%%%%%%%%%%%%%%%%%%%%%%%%%%%%%%%%%%%%%%%%%%%%%%%%%%%
  %%%%%%%%%%%%%%%%%%%%%%%%%%%%%%%%%%%%%%%%%%%%%%%%%%%%%%%%%%%%%%%%%%%%%%%%%%%%%%%%
  %%%%%%%%%%%%%%%%%%%%%%%%%%%%%%%%%%%%%%%%%%%%%%%%%%%%%%%%%%%%%%%%%%%%%%%%%%%%%%%%
  \node<2> (l') at (0,0) {
    \begin{minipage}{2.85cm}
      \inputminted[bgcolor=codebg, fontsize=\scriptsize]{c} {code/echo.rec}
    \end{minipage}};


  \node<2>[label={[xshift=2.05cm, yshift=-0.7cm]\LARGE \textbf{0}}] (r') at (3.9, 0) {
    \begin{minipage}{4.1cm}
      \inputminted[bgcolor=codebg, fontsize=\scriptsize]{coq} {code/getc.spec}
    \end{minipage}};

  \node<2>[label={[xshift=2.1cm, yshift=-0.7cm]\LARGE \textbf{0}}] (spec') at (8.8,0){
    \begin{minipage}{4.2cm}
      \inputminted[bgcolor=codebg, fontsize=\scriptsize]{coq} {code/echo.spec}
    \end{minipage}};
  %%%%%%%%%%%%%%%%%%%%%%%%%%%%%%%%%%%%%%%%%%%%%%%%%%%%%%%%%%%%%%%%%%%%%%%%%%%%%%%%
  %%%%%%%%%%%%%%%%%%%%%%%%%%%%%%%%%%%%%%%%%%%%%%%%%%%%%%%%%%%%%%%%%%%%%%%%%%%%%%%%
  %%%%%%%%%%%%%%%%%%%%%%%%%%%%%%%%%%%%%%%%%%%%%%%%%%%%%%%%%%%%%%%%%%%%%%%%%%%%%%%%
  \node<3> (l') at (0,0) {
    \begin{minipage}{2.85cm}
      \inputminted[bgcolor=codebg, fontsize=\scriptsize]{c} {code/echo.rec}
    \end{minipage}};

  \node<3>[label={[xshift=2.05cm, yshift=-0.7cm]\LARGE \textbf{0}}] (r') at (3.9, 0) {
    \begin{minipage}{4.1cm}
      \inputminted[bgcolor=codebg, fontsize=\scriptsize]{coq} {code/getc.spec}
    \end{minipage}};

  \node<3>[label={[xshift=2.1cm, yshift=-0.7cm, opacity = .5]\LARGE \textbf{0}}, opacity = .5] (spec') at (8.8,0){
    \begin{minipage}{4.2cm}
      \inputminted[bgcolor=codebg, fontsize=\scriptsize]{coq} {code/echo.spec}
    \end{minipage}};
  \node<3>[single arrow, draw=black, very thick, fill=yellow, rotate=-90,
      minimum width = 5pt, single arrow head extend=2pt,
      minimum height=7.5mm, label={[above=.3cm]\texttt{(Call f vs h)}}] at (1.63, 2.5){}; % length of arrow
  %%%%%%%%%%%%%%%%%%%%%%%%%%%%%%%%%%%%%%%%%%%%%%%%%%%%%%%%%%%%%%%%%%%%%%%%%%%%%%%%
  %%%%%%%%%%%%%%%%%%%%%%%%%%%%%%%%%%%%%%%%%%%%%%%%%%%%%%%%%%%%%%%%%%%%%%%%%%%%%%%%
  %%%%%%%%%%%%%%%%%%%%%%%%%%%%%%%%%%%%%%%%%%%%%%%%%%%%%%%%%%%%%%%%%%%%%%%%%%%%%%%%
  \node<4>[opacity =.5] (l') at (0,0) {
    \begin{minipage}{2.85cm}
      \inputminted[bgcolor=codebg, fontsize=\scriptsize]{c} {code/echo.rec}
    \end{minipage}};

  \node<4>[label={[xshift=2.05cm, yshift=-0.7cm, opacity=.5]\LARGE \textbf{0}}, opacity=.5] (r') at (3.9, 0) {
    \begin{minipage}{4.1cm}
      \inputminted[bgcolor=codebg, fontsize=\scriptsize]{coq} {code/getc.spec}
    \end{minipage}};

  \node<4>[label={[xshift=2.1cm, yshift=-0.7cm]\LARGE \textbf{0}}] (spec') at (8.8,0){
    \begin{minipage}{4.2cm}
      \inputminted[bgcolor=codebg, fontsize=\scriptsize]{coq} {code/echo.spec}
    \end{minipage}};

  \node<4-6>[single arrow, draw=black, very thick, fill=yellow, rotate=-90, opacity=0,
      minimum width = 5pt, single arrow head extend=2pt,
      minimum height=7.5mm, label={[opacity=.5, above=.3cm]\texttt{Call f vs h}}] at (1.63, 2.5){}; % length of arrow

  \node<4>[single arrow, draw=black, very thick, fill=yellow, rotate=-90,
      minimum width = 5pt, single arrow head extend=2pt,
      minimum height=7.5mm, label={[above=.3cm]\texttt{(Call f vs h)}}] at (8.8, 2.5){}; % length of arrow
  %%%%%%%%%%%%%%%%%%%%%%%%%%%%%%%%%%%%%%%%%%%%%%%%%%%%%%%%%%%%%%%%%%%%%%%%%%%%%%%%
  %%%%%%%%%%%%%%%%%%%%%%%%%%%%%%%%%%%%%%%%%%%%%%%%%%%%%%%%%%%%%%%%%%%%%%%%%%%%%%%%
  %%%%%%%%%%%%%%%%%%%%%%%%%%%%%%%%%%%%%%%%%%%%%%%%%%%%%%%%%%%%%%%%%%%%%%%%%%%%%%%%
  \node<5-6>[opacity =.5] (l') at (0,0) {
    \begin{minipage}{2.85cm}
      \inputminted[bgcolor=codebg, fontsize=\scriptsize]{c} {code/echo.rec}
    \end{minipage}};

  \node<5-6>[label={[xshift=2.05cm, yshift=-0.7cm, opacity=.5]\LARGE \textbf{0}}, opacity=.5] (r') at (3.9, 0) {
    \begin{minipage}{4.1cm}
      \inputminted[bgcolor=codebg, fontsize=\scriptsize]{coq} {code/getc.spec}
    \end{minipage}};

  \node<5>[label={[xshift=2.1cm, yshift=-0.7cm]\LARGE \textbf{0}}] (spec') at (8.8,0){
    \begin{minipage}{4.2cm}
      \inputminted[bgcolor=codebg, fontsize=\scriptsize, highlightlines={2}, highlightcolor=yellow!60]{coq} {code/echo.spec}
    \end{minipage}};

  \node<5-6>[single arrow, draw=black, very thick, fill=yellow, rotate=-90,
      minimum width = 5pt, single arrow head extend=2pt,
      minimum height=7.5mm, label={[above=.3cm]\texttt{(Call f vs h)}}] at (8.8, 2.5){}; % length of arrow
  %%%%%%%%%%%%%%%%%%%%%%%%%%%%%%%%%%%%%%%%%%%%%%%%%%%%%%%%%%%%%%%%%%%%%%%%%%%%%%%%
  %%%%%%%%%%%%%%%%%%%%%%%%%%%%%%%%%%%%%%%%%%%%%%%%%%%%%%%%%%%%%%%%%%%%%%%%%%%%%%%%
  %%%%%%%%%%%%%%%%%%%%%%%%%%%%%%%%%%%%%%%%%%%%%%%%%%%%%%%%%%%%%%%%%%%%%%%%%%%%%%%%
  \node<6>[label={[xshift=2.1cm, yshift=-0.7cm]\LARGE \textbf{0}}] (spec') at (8.8,0){
    \begin{minipage}{4.2cm}
      \inputminted[bgcolor=codebg, fontsize=\scriptsize, highlightlines={3}, highlightcolor=yellow!60]{coq} {code/echo.spec}
    \end{minipage}};
  %%%%%%%%%%%%%%%%%%%%%%%%%%%%%%%%%%%%%%%%%%%%%%%%%%%%%%%%%%%%%%%%%%%%%%%%%%%%%%%%
  %%%%%%%%%%%%%%%%%%%%%%%%%%%%%%%%%%%%%%%%%%%%%%%%%%%%%%%%%%%%%%%%%%%%%%%%%%%%%%%%
  %%%%%%%%%%%%%%%%%%%%%%%%%%%%%%%%%%%%%%%%%%%%%%%%%%%%%%%%%%%%%%%%%%%%%%%%%%%%%%%%
  \node<7-8>[opacity =.5] (l') at (0,0) {
    \begin{minipage}{2.85cm}
      \inputminted[bgcolor=codebg, fontsize=\scriptsize]{c} {code/echo.rec}
    \end{minipage}};

  \node<7-8>[label={[xshift=2.05cm, yshift=-0.7cm, opacity=.5]\LARGE \textbf{0}}, opacity=.5] (r') at (3.9, 0) {
    \begin{minipage}{4.1cm}
      \inputminted[bgcolor=codebg, fontsize=\scriptsize]{coq} {code/getc.spec}
    \end{minipage}};

  \node<7>[label={[xshift=2.1cm, yshift=-0.7cm]\LARGE \textbf{0}}] (spec') at (8.8,0){
    \begin{minipage}{4.2cm}
      \inputminted[bgcolor=codebg, fontsize=\scriptsize, highlightlines={4}, highlightcolor=yellow!60]{coq} {code/echo.spec}
    \end{minipage}};

  \node<7>[single arrow, draw=black, very thick, fill=yellow, rotate=-90, opacity=0,
      minimum width = 5pt, single arrow head extend=2pt,
      minimum height=7.5mm, label={[opacity=.5, above=.3cm]\texttt{(Call "echo" vs h)}}] at (1.63, 2.5){}; % length of arrow
  %%%%%%%%%%%%%%%%%%%%%%%%%%%%%%%%%%%%%%%%%%%%%%%%%%%%%%%%%%%%%%%%%%%%%%%%%%%%%%%%
  %%%%%%%%%%%%%%%%%%%%%%%%%%%%%%%%%%%%%%%%%%%%%%%%%%%%%%%%%%%%%%%%%%%%%%%%%%%%%%%%
  %%%%%%%%%%%%%%%%%%%%%%%%%%%%%%%%%%%%%%%%%%%%%%%%%%%%%%%%%%%%%%%%%%%%%%%%%%%%%%%%
  \node<8>[label={[xshift=2.1cm, yshift=-0.7cm]\LARGE \textbf{0}}] (spec') at (8.8,0){
    \begin{minipage}{4.2cm}
      \inputminted[bgcolor=codebg, fontsize=\scriptsize, highlightlines={5}, highlightcolor=yellow!60]{coq} {code/echo.spec}
    \end{minipage}};

  \node<8>[single arrow, draw=black, very thick, fill=yellow, rotate=-90, opacity=0,
      minimum width = 5pt, single arrow head extend=2pt,
      minimum height=7.5mm, label={[opacity=.5, above=.3cm]\texttt{(Call "echo" [] h)}}] at (1.63, 2.5){}; % length of arrow
  %%%%%%%%%%%%%%%%%%%%%%%%%%%%%%%%%%%%%%%%%%%%%%%%%%%%%%%%%%%%%%%%%%%%%%%%%%%%%%%%
  %%%%%%%%%%%%%%%%%%%%%%%%%%%%%%%%%%%%%%%%%%%%%%%%%%%%%%%%%%%%%%%%%%%%%%%%%%%%%%%%
  %%%%%%%%%%%%%%%%%%%%%%%%%%%%%%%%%%%%%%%%%%%%%%%%%%%%%%%%%%%%%%%%%%%%%%%%%%%%%%%%
  \node<9>[opacity =1] (l') at (0,0) {
    \begin{minipage}{2.85cm}
      \inputminted[bgcolor=codebg, fontsize=\scriptsize, highlightlines={1}, highlightcolor=yellow!60]{c} {code/echo.rec}
    \end{minipage}};

  \node<9-10>[label={[xshift=2.05cm, yshift=-0.7cm, opacity=.5]\LARGE \textbf{0}}, opacity=.5] (r') at (3.9, 0) {
    \begin{minipage}{4.1cm}
      \inputminted[bgcolor=codebg, fontsize=\scriptsize]{coq} {code/getc.spec}
    \end{minipage}};

  \node<9-10>[label={[xshift=2.1cm, yshift=-0.7cm, opacity=.5]\LARGE \textbf{0}}, opacity=.5] (spec') at (8.8,0){
    \begin{minipage}{4.2cm}
      \inputminted[bgcolor=codebg, fontsize=\scriptsize, highlightlines={6}, highlightcolor=yellow!60]{coq} {code/echo.spec}
    \end{minipage}};
  %%%%%%%%%%%%%%%%%%%%%%%%%%%%%%%%%%%%%%%%%%%%%%%%%%%%%%%%%%%%%%%%%%%%%%%%%%%%%%%%
  %%%%%%%%%%%%%%%%%%%%%%%%%%%%%%%%%%%%%%%%%%%%%%%%%%%%%%%%%%%%%%%%%%%%%%%%%%%%%%%%
  %%%%%%%%%%%%%%%%%%%%%%%%%%%%%%%%%%%%%%%%%%%%%%%%%%%%%%%%%%%%%%%%%%%%%%%%%%%%%%%%
  \node<10>[opacity =1] (l') at (0,0) {
    \begin{minipage}{2.85cm}
      \inputminted[bgcolor=codebg, fontsize=\scriptsize, highlightlines={2}, highlightcolor=yellow!60]{c} {code/echo.rec}
    \end{minipage}};
  %%%%%%%%%%%%%%%%%%%%%%%%%%%%%%%%%%%%%%%%%%%%%%%%%%%%%%%%%%%%%%%%%%%%%%%%%%%%%%%%
  %%%%%%%%%%%%%%%%%%%%%%%%%%%%%%%%%%%%%%%%%%%%%%%%%%%%%%%%%%%%%%%%%%%%%%%%%%%%%%%%
  %%%%%%%%%%%%%%%%%%%%%%%%%%%%%%%%%%%%%%%%%%%%%%%%%%%%%%%%%%%%%%%%%%%%%%%%%%%%%%%%

  \node<11>[opacity =.5] (l') at (0,0) {
    \begin{minipage}{2.85cm}
      \inputminted[bgcolor=codebg, fontsize=\scriptsize, highlightlines={2}, highlightcolor=yellow!60]{c} {code/echo.rec}
    \end{minipage}};

  \node<11>[label={[xshift=2.05cm, yshift=-0.7cm, opacity=.5]\LARGE \textbf{0}}, opacity=.5] (r') at (3.9, 0) {
    \begin{minipage}{4.1cm}
      \inputminted[bgcolor=codebg, fontsize=\scriptsize]{coq} {code/getc.spec}
    \end{minipage}};

  \node<11-24>[label={[xshift=2.1cm, yshift=-0.7cm, opacity=.5]\LARGE \textbf{0}}, opacity=.5] (spec') at (8.8,0){
    \begin{minipage}{4.2cm}
      \inputminted[bgcolor=codebg, fontsize=\scriptsize, highlightlines={6}, highlightcolor=yellow!60]{coq} {code/echo.spec}
    \end{minipage}};

  \node<11>[single arrow, draw=black, very thick, fill=yellow, rotate=90, opacity=1,
      minimum width = 5pt, single arrow head extend=2pt,
      minimum height=7.5mm, label={[opacity=1, above=.3cm]\texttt{(Call "getc" [] h)}}] at (0, 2.5){}; % length of arrow

  %%%%%%%%%%%%%%%%%%%%%%%%%%%%%%%%%%%%%%%%%%%%%%%%%%%%%%%%%%%%%%%%%%%%%%%%%%%%%%%%
  %%%%%%%%%%%%%%%%%%%%%%%%%%%%%%%%%%%%%%%%%%%%%%%%%%%%%%%%%%%%%%%%%%%%%%%%%%%%%%%%
  %%%%%%%%%%%%%%%%%%%%%%%%%%%%%%%%%%%%%%%%%%%%%%%%%%%%%%%%%%%%%%%%%%%%%%%%%%%%%%%%

  \node<12-21>[opacity =.5] (l') at (0,0) {
    \begin{minipage}{2.85cm}
      \inputminted[bgcolor=codebg, fontsize=\scriptsize, highlightlines={2}, highlightcolor=yellow!60]{c} {code/echo.rec}
    \end{minipage}};

  \node<12>[label={[xshift=2.05cm, yshift=-0.7cm, opacity=1]\LARGE \textbf{0}}, opacity=1] (r') at (3.9, 0) {
    \begin{minipage}{4.1cm}
      \inputminted[bgcolor=codebg, fontsize=\scriptsize]{coq} {code/getc.spec}
    \end{minipage}};

  \node<12-15>[single arrow, draw=black, very thick, fill=yellow, rotate=-90, opacity=1,
      minimum width = 5pt, single arrow head extend=2pt,
      minimum height=7.5mm, label={[opacity=1, above=.3cm]\texttt{(Call "getc" [] h)}}] at (3.9, 2.5){}; % length of arrow

  %%%%%%%%%%%%%%%%%%%%%%%%%%%%%%%%%%%%%%%%%%%%%%%%%%%%%%%%%%%%%%%%%%%%%%%%%%%%%%%%
  %%%%%%%%%%%%%%%%%%%%%%%%%%%%%%%%%%%%%%%%%%%%%%%%%%%%%%%%%%%%%%%%%%%%%%%%%%%%%%%%
  %%%%%%%%%%%%%%%%%%%%%%%%%%%%%%%%%%%%%%%%%%%%%%%%%%%%%%%%%%%%%%%%%%%%%%%%%%%%%%%%

  \node<13>[label={[xshift=2.05cm, yshift=-0.7cm, opacity=1]\LARGE \textbf{0}}, opacity=1] (r') at (3.9, 0) {
    \begin{minipage}{4.1cm}
      \inputminted[bgcolor=codebg, fontsize=\scriptsize, highlightlines={2}, highlightcolor=yellow!60]{coq} {code/getc.spec}
    \end{minipage}};

  %%%%%%%%%%%%%%%%%%%%%%%%%%%%%%%%%%%%%%%%%%%%%%%%%%%%%%%%%%%%%%%%%%%%%%%%%%%%%%%%
  %%%%%%%%%%%%%%%%%%%%%%%%%%%%%%%%%%%%%%%%%%%%%%%%%%%%%%%%%%%%%%%%%%%%%%%%%%%%%%%%
  %%%%%%%%%%%%%%%%%%%%%%%%%%%%%%%%%%%%%%%%%%%%%%%%%%%%%%%%%%%%%%%%%%%%%%%%%%%%%%%%

  \node<14>[label={[xshift=2.05cm, yshift=-0.7cm, opacity=1]\LARGE \textbf{0}}, opacity=1] (r') at (3.9, 0) {
    \begin{minipage}{4.1cm}
      \inputminted[bgcolor=codebg, fontsize=\scriptsize, highlightlines={3}, highlightcolor=yellow!60]{coq} {code/getc.spec}
    \end{minipage}};

  %%%%%%%%%%%%%%%%%%%%%%%%%%%%%%%%%%%%%%%%%%%%%%%%%%%%%%%%%%%%%%%%%%%%%%%%%%%%%%%%
  %%%%%%%%%%%%%%%%%%%%%%%%%%%%%%%%%%%%%%%%%%%%%%%%%%%%%%%%%%%%%%%%%%%%%%%%%%%%%%%%
  %%%%%%%%%%%%%%%%%%%%%%%%%%%%%%%%%%%%%%%%%%%%%%%%%%%%%%%%%%%%%%%%%%%%%%%%%%%%%%%%
  \node<15>[label={[xshift=2.05cm, yshift=-0.7cm, opacity=1]\LARGE \textbf{0}}, opacity=1] (r') at (3.9, 0) {
    \begin{minipage}{4.1cm}
      \inputminted[bgcolor=codebg, fontsize=\scriptsize, highlightlines={4}, highlightcolor=yellow!60]{coq} {code/getc.spec}
    \end{minipage}};

  %%%%%%%%%%%%%%%%%%%%%%%%%%%%%%%%%%%%%%%%%%%%%%%%%%%%%%%%%%%%%%%%%%%%%%%%%%%%%%%%
  %%%%%%%%%%%%%%%%%%%%%%%%%%%%%%%%%%%%%%%%%%%%%%%%%%%%%%%%%%%%%%%%%%%%%%%%%%%%%%%%
  %%%%%%%%%%%%%%%%%%%%%%%%%%%%%%%%%%%%%%%%%%%%%%%%%%%%%%%%%%%%%%%%%%%%%%%%%%%%%%%%
  \node<16>[label={[xshift=2.05cm, yshift=-0.7cm, opacity=1]\LARGE \textbf{0}}, opacity=1] (r') at (3.9, 0) {
    \begin{minipage}{4.1cm}
      \inputminted[bgcolor=codebg, fontsize=\scriptsize, highlightlines={5}, highlightcolor=yellow!60]{coq} {code/getc.spec}
    \end{minipage}};

  \node<16-17>[single arrow, draw=black, very thick, fill=yellow, rotate=-90, opacity=0,
      minimum width = 5pt, single arrow head extend=2pt,
      minimum height=7.5mm, label={[opacity=1, above=.3cm]\texttt{(Call "getc" [] h)}}] at (3.9, 2.5){}; % length of arrow
  %%%%%%%%%%%%%%%%%%%%%%%%%%%%%%%%%%%%%%%%%%%%%%%%%%%%%%%%%%%%%%%%%%%%%%%%%%%%%%%%
  %%%%%%%%%%%%%%%%%%%%%%%%%%%%%%%%%%%%%%%%%%%%%%%%%%%%%%%%%%%%%%%%%%%%%%%%%%%%%%%%
  %%%%%%%%%%%%%%%%%%%%%%%%%%%%%%%%%%%%%%%%%%%%%%%%%%%%%%%%%%%%%%%%%%%%%%%%%%%%%%%%
  \node<17>[label={[xshift=2.05cm, yshift=-0.7cm, opacity=1]\LARGE \textbf{0}}, opacity=1] (r') at (3.9, 0) {
    \begin{minipage}{4.1cm}
      \inputminted[bgcolor=codebg, fontsize=\scriptsize, highlightlines={6}, highlightcolor=yellow!60]{coq} {code/getc.spec}
    \end{minipage}};
  %%%%%%%%%%%%%%%%%%%%%%%%%%%%%%%%%%%%%%%%%%%%%%%%%%%%%%%%%%%%%%%%%%%%%%%%%%%%%%%%
  %%%%%%%%%%%%%%%%%%%%%%%%%%%%%%%%%%%%%%%%%%%%%%%%%%%%%%%%%%%%%%%%%%%%%%%%%%%%%%%%
  %%%%%%%%%%%%%%%%%%%%%%%%%%%%%%%%%%%%%%%%%%%%%%%%%%%%%%%%%%%%%%%%%%%%%%%%%%%%%%%%
  \node<18>[label={[xshift=2.05cm, yshift=-0.7cm, opacity=1]\LARGE \textbf{0}}, opacity=1] (r') at (3.9, 0) {
    \begin{minipage}{4.1cm}
      \inputminted[bgcolor=codebg, fontsize=\scriptsize, highlightlines={7}, highlightcolor=yellow!60]{coq} {code/getc.spec}
    \end{minipage}};
  %%%%%%%%%%%%%%%%%%%%%%%%%%%%%%%%%%%%%%%%%%%%%%%%%%%%%%%%%%%%%%%%%%%%%%%%%%%%%%%%
  %%%%%%%%%%%%%%%%%%%%%%%%%%%%%%%%%%%%%%%%%%%%%%%%%%%%%%%%%%%%%%%%%%%%%%%%%%%%%%%%
  %%%%%%%%%%%%%%%%%%%%%%%%%%%%%%%%%%%%%%%%%%%%%%%%%%%%%%%%%%%%%%%%%%%%%%%%%%%%%%%%
  \node<19>[label={[xshift=2.05cm, yshift=-0.7cm, opacity=1]\LARGE \textbf{1}}, opacity=1] (r') at (3.9, 0) {
    \begin{minipage}{4.1cm}
      \inputminted[bgcolor=codebg, fontsize=\scriptsize, highlightlines={8}, highlightcolor=yellow!60]{coq} {code/getc.spec}
    \end{minipage}};
  %%%%%%%%%%%%%%%%%%%%%%%%%%%%%%%%%%%%%%%%%%%%%%%%%%%%%%%%%%%%%%%%%%%%%%%%%%%%%%%%
  %%%%%%%%%%%%%%%%%%%%%%%%%%%%%%%%%%%%%%%%%%%%%%%%%%%%%%%%%%%%%%%%%%%%%%%%%%%%%%%%
  %%%%%%%%%%%%%%%%%%%%%%%%%%%%%%%%%%%%%%%%%%%%%%%%%%%%%%%%%%%%%%%%%%%%%%%%%%%%%%%%
  \node<20>[label={[xshift=2.05cm, yshift=-0.7cm, opacity=1]\LARGE \textbf{1}}, opacity=1] (r') at (3.9, 0) {
    \begin{minipage}{4.1cm}
      \inputminted[bgcolor=codebg, fontsize=\scriptsize, highlightlines={9}, highlightcolor=yellow!60]{coq} {code/getc.spec}
    \end{minipage}};

  \node<20>[single arrow, draw=black, very thick, fill=yellow, rotate=90, opacity=1,
      minimum width = 5pt, single arrow head extend=2pt,
      minimum height=7.5mm, label={[opacity=1, above=.3cm]\texttt{(Return 0 h)}}] at (3.9, 2.5){}; % length of arrow
  %%%%%%%%%%%%%%%%%%%%%%%%%%%%%%%%%%%%%%%%%%%%%%%%%%%%%%%%%%%%%%%%%%%%%%%%%%%%%%%%
  %%%%%%%%%%%%%%%%%%%%%%%%%%%%%%%%%%%%%%%%%%%%%%%%%%%%%%%%%%%%%%%%%%%%%%%%%%%%%%%%
  %%%%%%%%%%%%%%%%%%%%%%%%%%%%%%%%%%%%%%%%%%%%%%%%%%%%%%%%%%%%%%%%%%%%%%%%%%%%%%%%
  \node<21->[label={[xshift=2.05cm, yshift=-0.7cm, opacity=.5]\LARGE \textbf{1}}, opacity=.5] (r') at (3.9, 0) {
    \begin{minipage}{4.1cm}
      \inputminted[bgcolor=codebg, fontsize=\scriptsize, highlightlines={2}, highlightcolor=yellow!60]{coq} {code/getc.spec}
    \end{minipage}};

  \node<21>[single arrow, draw=black, very thick, fill=yellow, rotate=90, opacity=1,
      minimum width = 5pt, single arrow head extend=2pt,
      minimum height=7.5mm, label={[opacity=1, above=.3cm]\texttt{(Return 0 h)}}] at (3.9, 2.5){}; % length of arrow
  %%%%%%%%%%%%%%%%%%%%%%%%%%%%%%%%%%%%%%%%%%%%%%%%%%%%%%%%%%%%%%%%%%%%%%%%%%%%%%%%
  %%%%%%%%%%%%%%%%%%%%%%%%%%%%%%%%%%%%%%%%%%%%%%%%%%%%%%%%%%%%%%%%%%%%%%%%%%%%%%%%
  %%%%%%%%%%%%%%%%%%%%%%%%%%%%%%%%%%%%%%%%%%%%%%%%%%%%%%%%%%%%%%%%%%%%%%%%%%%%%%%%

  \node<22>[opacity =1] (l') at (0,0) {
    \begin{minipage}{2.85cm}
      \inputminted[bgcolor=codebg, fontsize=\scriptsize, highlightlines={2}, highlightcolor=yellow!60]{c} {code/echo.rec}
    \end{minipage}};

  \node<22>[single arrow, draw=black, very thick, fill=yellow, rotate=-90, opacity=1,
      minimum width = 5pt, single arrow head extend=2pt,
      minimum height=7.5mm, label={[opacity=1, above=.3cm]\texttt{(Return 0 h)}}] at (0, 2.5){}; % length of arrow

  %%%%%%%%%%%%%%%%%%%%%%%%%%%%%%%%%%%%%%%%%%%%%%%%%%%%%%%%%%%%%%%%%%%%%%%%%%%%%%%%
  %%%%%%%%%%%%%%%%%%%%%%%%%%%%%%%%%%%%%%%%%%%%%%%%%%%%%%%%%%%%%%%%%%%%%%%%%%%%%%%%
  %%%%%%%%%%%%%%%%%%%%%%%%%%%%%%%%%%%%%%%%%%%%%%%%%%%%%%%%%%%%%%%%%%%%%%%%%%%%%%%%

  \node<23>[opacity =1] (l') at (0,0) {
    \begin{minipage}{2.85cm}
      \inputminted[bgcolor=codebg, fontsize=\scriptsize, highlightlines={3}, highlightcolor=yellow!60]{c} {code/echo.rec}
    \end{minipage}};

  \node<23>[single arrow, draw=black, very thick, fill=yellow, rotate=90, opacity=1,
      minimum width = 5pt, single arrow head extend=2pt,
      minimum height=7.5mm, label={[opacity=1, above=.3cm]\texttt{(Call "putc" 0 h)}}] at (0, 2.5){}; % length of arrow

  %%%%%%%%%%%%%%%%%%%%%%%%%%%%%%%%%%%%%%%%%%%%%%%%%%%%%%%%%%%%%%%%%%%%%%%%%%%%%%%%
  %%%%%%%%%%%%%%%%%%%%%%%%%%%%%%%%%%%%%%%%%%%%%%%%%%%%%%%%%%%%%%%%%%%%%%%%%%%%%%%%
  %%%%%%%%%%%%%%%%%%%%%%%%%%%%%%%%%%%%%%%%%%%%%%%%%%%%%%%%%%%%%%%%%%%%%%%%%%%%%%%%

  \node<24>[opacity =.5] (l') at (0,0) {
    \begin{minipage}{2.85cm}
      \inputminted[bgcolor=codebg, fontsize=\scriptsize, highlightlines={3}, highlightcolor=yellow!60]{c} {code/echo.rec}
    \end{minipage}};

  \node<24>[single arrow, draw=black, very thick, fill=yellow, rotate=90, opacity=1,
      minimum width = 5pt, single arrow head extend=2pt,
      minimum height=7.5mm, label={[opacity=1, above=.3cm]\texttt{(Call "putc" 0 h)}}] at (1.63, 2.5){}; % length of arrow

  %%%%%%%%%%%%%%%%%%%%%%%%%%%%%%%%%%%%%%%%%%%%%%%%%%%%%%%%%%%%%%%%%%%%%%%%%%%%%%%%
  %%%%%%%%%%%%%%%%%%%%%%%%%%%%%%%%%%%%%%%%%%%%%%%%%%%%%%%%%%%%%%%%%%%%%%%%%%%%%%%%
  %%%%%%%%%%%%%%%%%%%%%%%%%%%%%%%%%%%%%%%%%%%%%%%%%%%%%%%%%%%%%%%%%%%%%%%%%%%%%%%%

  \node<25-31>[opacity =.5] (l') at (0,0) {
    \begin{minipage}{2.85cm}
      \inputminted[bgcolor=codebg, fontsize=\scriptsize, highlightlines={3}, highlightcolor=yellow!60]{c} {code/echo.rec}
    \end{minipage}};

  \node<25>[label={[xshift=2.1cm, yshift=-0.7cm]\LARGE \textbf{0}}] (spec') at (8.8,0){
    \begin{minipage}{4.2cm}
      \inputminted[bgcolor=codebg, fontsize=\scriptsize, highlightlines={6}, highlightcolor=yellow!60]{coq} {code/echo.spec}
    \end{minipage}};

  \node<25-29>[single arrow, draw=black, very thick, fill=yellow, rotate=-90, opacity=1,
      minimum width = 5pt, single arrow head extend=2pt,
      minimum height=7.5mm, label={[opacity=1, above=.3cm]\texttt{(Call "putc" 0 h)}}] at (8.8, 2.5){}; % length of arrow

  %%%%%%%%%%%%%%%%%%%%%%%%%%%%%%%%%%%%%%%%%%%%%%%%%%%%%%%%%%%%%%%%%%%%%%%%%%%%%%%%
  %%%%%%%%%%%%%%%%%%%%%%%%%%%%%%%%%%%%%%%%%%%%%%%%%%%%%%%%%%%%%%%%%%%%%%%%%%%%%%%%
  %%%%%%%%%%%%%%%%%%%%%%%%%%%%%%%%%%%%%%%%%%%%%%%%%%%%%%%%%%%%%%%%%%%%%%%%%%%%%%%%

  \node<26>[label={[xshift=2.1cm, yshift=-0.7cm]\LARGE \textbf{0}}] (spec') at (8.8,0){
    \begin{minipage}{4.2cm}
      \inputminted[bgcolor=codebg, fontsize=\scriptsize, highlightlines={6}, highlightcolor=yellow!60]{coq} {code/echo.spec}
    \end{minipage}};

  %%%%%%%%%%%%%%%%%%%%%%%%%%%%%%%%%%%%%%%%%%%%%%%%%%%%%%%%%%%%%%%%%%%%%%%%%%%%%%%%
  %%%%%%%%%%%%%%%%%%%%%%%%%%%%%%%%%%%%%%%%%%%%%%%%%%%%%%%%%%%%%%%%%%%%%%%%%%%%%%%%
  %%%%%%%%%%%%%%%%%%%%%%%%%%%%%%%%%%%%%%%%%%%%%%%%%%%%%%%%%%%%%%%%%%%%%%%%%%%%%%%%

  \node<27>[label={[xshift=2.1cm, yshift=-0.7cm]\LARGE \textbf{1}}] (spec') at (8.8,0){
    \begin{minipage}{4.2cm}
      \inputminted[bgcolor=codebg, fontsize=\scriptsize, highlightlines={7}, highlightcolor=yellow!60]{coq} {code/echo.spec}
    \end{minipage}};
  %%%%%%%%%%%%%%%%%%%%%%%%%%%%%%%%%%%%%%%%%%%%%%%%%%%%%%%%%%%%%%%%%%%%%%%%%%%%%%%%
  %%%%%%%%%%%%%%%%%%%%%%%%%%%%%%%%%%%%%%%%%%%%%%%%%%%%%%%%%%%%%%%%%%%%%%%%%%%%%%%%
  %%%%%%%%%%%%%%%%%%%%%%%%%%%%%%%%%%%%%%%%%%%%%%%%%%%%%%%%%%%%%%%%%%%%%%%%%%%%%%%%

  \node<28>[label={[xshift=2.1cm, yshift=-0.7cm]\LARGE \textbf{1}}] (spec') at (8.8,0){
    \begin{minipage}{4.2cm}
      \inputminted[bgcolor=codebg, fontsize=\scriptsize, highlightlines={7}, highlightcolor=yellow!60]{coq} {code/echo.spec}
    \end{minipage}};
  %%%%%%%%%%%%%%%%%%%%%%%%%%%%%%%%%%%%%%%%%%%%%%%%%%%%%%%%%%%%%%%%%%%%%%%%%%%%%%%%
  %%%%%%%%%%%%%%%%%%%%%%%%%%%%%%%%%%%%%%%%%%%%%%%%%%%%%%%%%%%%%%%%%%%%%%%%%%%%%%%%
  %%%%%%%%%%%%%%%%%%%%%%%%%%%%%%%%%%%%%%%%%%%%%%%%%%%%%%%%%%%%%%%%%%%%%%%%%%%%%%%%

  \node<29-30>[label={[xshift=2.1cm, yshift=-0.7cm]\LARGE \textbf{1}}] (spec') at (8.8,0){
    \begin{minipage}{4.2cm}
      \inputminted[bgcolor=codebg, fontsize=\scriptsize, highlightlines={8}, highlightcolor=yellow!60]{coq} {code/echo.spec}
    \end{minipage}};

  %%%%%%%%%%%%%%%%%%%%%%%%%%%%%%%%%%%%%%%%%%%%%%%%%%%%%%%%%%%%%%%%%%%%%%%%%%%%%%%%
  %%%%%%%%%%%%%%%%%%%%%%%%%%%%%%%%%%%%%%%%%%%%%%%%%%%%%%%%%%%%%%%%%%%%%%%%%%%%%%%%
  %%%%%%%%%%%%%%%%%%%%%%%%%%%%%%%%%%%%%%%%%%%%%%%%%%%%%%%%%%%%%%%%%%%%%%%%%%%%%%%%

  \node<30>[single arrow, draw=black, very thick, fill=yellow, rotate=90, opacity=1,
      minimum width = 5pt, single arrow head extend=2pt,
      minimum height=7.5mm, label={[opacity=1, above=.3cm]\texttt{(Return v h)}}] at (8.8, 2.5){}; % length of arrow

  %%%%%%%%%%%%%%%%%%%%%%%%%%%%%%%%%%%%%%%%%%%%%%%%%%%%%%%%%%%%%%%%%%%%%%%%%%%%%%%%
  %%%%%%%%%%%%%%%%%%%%%%%%%%%%%%%%%%%%%%%%%%%%%%%%%%%%%%%%%%%%%%%%%%%%%%%%%%%%%%%%
  %%%%%%%%%%%%%%%%%%%%%%%%%%%%%%%%%%%%%%%%%%%%%%%%%%%%%%%%%%%%%%%%%%%%%%%%%%%%%%%%

  \node<31-34>[label={[xshift=2.1cm, yshift=-0.7cm, opacity=.5]\LARGE \textbf{1}}, opacity=.5] (spec') at (8.8,0){
    \begin{minipage}{4.2cm}
      \inputminted[bgcolor=codebg, fontsize=\scriptsize, highlightlines={8}, highlightcolor=yellow!60]{coq} {code/echo.spec}
    \end{minipage}};

  \node<31>[single arrow, draw=black, very thick, fill=yellow, rotate=-90, opacity=1,
      minimum width = 5pt, single arrow head extend=2pt,
      minimum height=7.5mm, label={[opacity=1, above=.3cm]\texttt{(Return v h)}}] at (1.63, 2.5){}; % length of arrow

  %%%%%%%%%%%%%%%%%%%%%%%%%%%%%%%%%%%%%%%%%%%%%%%%%%%%%%%%%%%%%%%%%%%%%%%%%%%%%%%%
  %%%%%%%%%%%%%%%%%%%%%%%%%%%%%%%%%%%%%%%%%%%%%%%%%%%%%%%%%%%%%%%%%%%%%%%%%%%%%%%%
  %%%%%%%%%%%%%%%%%%%%%%%%%%%%%%%%%%%%%%%%%%%%%%%%%%%%%%%%%%%%%%%%%%%%%%%%%%%%%%%%
  \node<32>[opacity =1] (l') at (0,0) {
    \begin{minipage}{2.85cm}
      \inputminted[bgcolor=codebg, fontsize=\scriptsize, highlightlines={3}, highlightcolor=yellow!60]{c} {code/echo.rec}
    \end{minipage}};

  \node<32>[single arrow, draw=black, very thick, fill=yellow, rotate=-90, opacity=1,
      minimum width = 5pt, single arrow head extend=2pt,
      minimum height=7.5mm, label={[opacity=1, above=.3cm]\texttt{(Return v h)}}] at (0, 2.5){}; % length of arrow

  %%%%%%%%%%%%%%%%%%%%%%%%%%%%%%%%%%%%%%%%%%%%%%%%%%%%%%%%%%%%%%%%%%%%%%%%%%%%%%%%
  %%%%%%%%%%%%%%%%%%%%%%%%%%%%%%%%%%%%%%%%%%%%%%%%%%%%%%%%%%%%%%%%%%%%%%%%%%%%%%%%
  %%%%%%%%%%%%%%%%%%%%%%%%%%%%%%%%%%%%%%%%%%%%%%%%%%%%%%%%%%%%%%%%%%%%%%%%%%%%%%%%
  \node<33>[opacity =1] (l') at (0,0) {
    \begin{minipage}{2.85cm}
      \inputminted[bgcolor=codebg, fontsize=\scriptsize, highlightlines={4}, highlightcolor=yellow!60]{c} {code/echo.rec}
    \end{minipage}};

  \node<33>[single arrow, draw=black, very thick, fill=yellow, rotate=90, opacity=1,
      minimum width = 5pt, single arrow head extend=2pt,
      minimum height=7.5mm, label={[opacity=1, above=.3cm]\texttt{(Return 0 h)}}] at (0, 2.5){}; % length of arrow

  %%%%%%%%%%%%%%%%%%%%%%%%%%%%%%%%%%%%%%%%%%%%%%%%%%%%%%%%%%%%%%%%%%%%%%%%%%%%%%%%
  %%%%%%%%%%%%%%%%%%%%%%%%%%%%%%%%%%%%%%%%%%%%%%%%%%%%%%%%%%%%%%%%%%%%%%%%%%%%%%%%
  %%%%%%%%%%%%%%%%%%%%%%%%%%%%%%%%%%%%%%%%%%%%%%%%%%%%%%%%%%%%%%%%%%%%%%%%%%%%%%%%
  \node<34->[opacity =.5] (l') at (0,0) {
    \begin{minipage}{2.85cm}
      \inputminted[bgcolor=codebg, fontsize=\scriptsize, highlightlines={4}, highlightcolor=yellow!60]{c} {code/echo.rec}
    \end{minipage}};

  \node<34>[single arrow, draw=black, very thick, fill=yellow, rotate=90, opacity=1,
      minimum width = 5pt, single arrow head extend=2pt,
      minimum height=7.5mm, label={[opacity=1, above=.3cm]\texttt{(Return 0 h)}}] at (3.9, 2.5){}; % length of arrow

  %%%%%%%%%%%%%%%%%%%%%%%%%%%%%%%%%%%%%%%%%%%%%%%%%%%%%%%%%%%%%%%%%%%%%%%%%%%%%%%%
  %%%%%%%%%%%%%%%%%%%%%%%%%%%%%%%%%%%%%%%%%%%%%%%%%%%%%%%%%%%%%%%%%%%%%%%%%%%%%%%%
  %%%%%%%%%%%%%%%%%%%%%%%%%%%%%%%%%%%%%%%%%%%%%%%%%%%%%%%%%%%%%%%%%%%%%%%%%%%%%%%%

  \node<35>[single arrow, draw=black, very thick, fill=yellow, rotate=-90, opacity=1,
      minimum width = 5pt, single arrow head extend=2pt,
      minimum height=7.5mm, label={[opacity=1, above=.3cm]\texttt{(Return 0 h)}}] at (8.8, 2.5){}; % length of arrow

  \node<35>[label={[xshift=2.1cm, yshift=-0.7cm, opacity=1]\LARGE \textbf{1}}, opacity=1] (spec') at (8.8,0){
    \begin{minipage}{4.2cm}
      \inputminted[bgcolor=codebg, fontsize=\scriptsize, highlightlines={9}, highlightcolor=yellow!60]{coq} {code/echo.spec}
    \end{minipage}};

  %%%%%%%%%%%%%%%%%%%%%%%%%%%%%%%%%%%%%%%%%%%%%%%%%%%%%%%%%%%%%%%%%%%%%%%%%%%%%%%%
  %%%%%%%%%%%%%%%%%%%%%%%%%%%%%%%%%%%%%%%%%%%%%%%%%%%%%%%%%%%%%%%%%%%%%%%%%%%%%%%%
  %%%%%%%%%%%%%%%%%%%%%%%%%%%%%%%%%%%%%%%%%%%%%%%%%%%%%%%%%%%%%%%%%%%%%%%%%%%%%%%%

  \node<36>[label={[xshift=2.1cm, yshift=-0.7cm, opacity=1]\LARGE \textbf{1}}, opacity=1] (spec') at (8.8,0){
    \begin{minipage}{4.2cm}
      \inputminted[bgcolor=codebg, fontsize=\scriptsize, highlightlines={10}, highlightcolor=yellow!60]{coq} {code/echo.spec}
    \end{minipage}};

  \draw[rounded corners, thick] (-1.6,-2) rectangle (6.15, 3.5);
  \draw (current bounding box.north east) -- (current bounding box.north west) -- (current bounding box.south west) -- (current bounding box.south east) -- cycle;
\end{tikzpicture}

    \end{figure}
  \end{textblock}
\end{frame}

\section{RTL to LTL}
\begin{frame}[t]{RTL to LTL}

\begin{textblock}{18}[0,0](1,4)
\begin{block}{RTL Transition}
  \footnotesize
\[
  \inference{
    {\textcolor{black!60}{c(pc) = \lfloor \texttt{op}_{RTL}(op, \vec{x}, y, pc')\rfloor}}
      &{\texttt{eval\_op}(\_, \_,op, \textcolor{red}{M(\vec{x})}) = \lfloor v \rfloor}}
    {\_ \vdash S(\_, \_, \_, \textcolor{gray}{pc}, \textcolor{red}{M},\_) \xrightarrow{\epsilon} S(\_, \_, \_, pc', \textcolor{red}{M[y \gets v]}, \_)}
\]
\end{block}
\end{textblock}
%% We have a similar rule for LTL where the instruction $\texttt{op}_{LTL}$ operates on optional register pairs (\verb!rpair mreg!) and the state contains a map $L$ from locations (Stack slots and registers) to values. We extend the location map in a straightforward fashion by defining three operations: \verb!combine!, \verb!loword!, and \verb!hiword! to perform splitting and combining on the level of values.
%% %% \begin{align*}
%% %%   L((r1, r2)) &\stackrel{def}{=} \texttt{combine}(L(r1), L(r2)),\ and\\
%% %%   L[(r1, r2) \gets v] &\stackrel{def}{=} L[r1 \gets \texttt{hiword}(v)][r2 \gets \texttt{loword}(v)].
%% %% \end{align*}

%% Below, we show the transition rule for $\texttt{op}_{LTL}$, which is similar to that of RTL, but captures the execution in a basic block, and therefore does not increase the program counter. Note however, that the arguments $\vec{p}$ and result location $p$ are either single registers or pairs and are evaluated as given above.
\begin{textblock}{18}[0,0](1,10)
\begin{block}{LTL Transition}
  \vspace{-.4cm}
  \footnotesize
\[
  \inference{
    {\texttt{eval\_op}(\_, \_,\texttt{op}, \textcolor{red}{L(\vec{p})}) = \lfloor v \rfloor}}
    {\_ \vdash B(\_, \_, \_, \textcolor{black!60}{\texttt{op}_{LTL}(op, \vec{p}, q) :: bb},\textcolor{red}{L},\_) \xrightarrow{\epsilon} B(\_, \_, \_, bb, \textcolor{red}{L[q \gets v]}, \_)}
\]
\end{block}
\end{textblock}
\end{frame}

\section{Evaluation}
\begin{frame}{Results - Compile Time (Arm hard float)}
\begin{textblock}{18}[0,0](0.19,2)
\end{textblock}
\end{frame}

\definecolor{good}{RGB}{0, 179, 0}
\definecolor{nice}{RGB}{153, 255, 51}
\definecolor{bad}{RGB}{255, 153, 0}

\begin{frame}{Results - Code Size}
\begin{table}[h]
\centering
\begin{tabular}{ c|c|c|c |c| c}
   & vpr & mesa & fuzz1 & fuzz2 & fuzz3\\
 \hline\hline
 arm\_hard &\cellcolor{nice} -0.83\% & \cellcolor{nice}-1.77\% & \cellcolor{good}-4.78\% & \cellcolor{good}-4.7\%& \cellcolor{good}-4.7\%\\
  arm\_soft & -0.2\% & -0.71\% & -0.2\% & \cellcolor{bad}+0.19\%&\cellcolor{bad} +0.27\% \\
\end{tabular}
  \label{tab:eval:size}
\end{table}
\end{frame}


\begin{frame}{Contributions}
  \begin{itemize}
    \item Improved model of the Arm assembly semantics
    \item Proved all architectures correct w.o. changing their semantics
    \item New and more general register allocator
    \item Enable future support for TriCore architecture
    \item Small positive impact on code generation
  \end{itemize}
\end{frame}

\begin{frame}{Questions?}
\tableofcontents
\end{frame}

\begin{frame}[noframenumbering]{Results - Allocator Statistics}
  \footnotesize
\begin{table}[H]
  \centering
  \setlength{\tabcolsep}{0.495em}
\begin{tabular}{c
  |c !{\color{gray!30}\vrule width .5pt}c
  |c !{\color{gray!30} \vrule width .5pt}c
  |c !{\color{gray!30} \vrule width .5pt}c
  |c !{\color{gray!30} \vrule width .5pt}c}
  \multicolumn{1}{c}{}&  \multicolumn{2}{c}{Remaining} & \multicolumn{2}{c}{Inserted} &\multicolumn{2}{c}{Reloads} &\multicolumn{2}{c}{Spills}\\
  & C & C${}^{p}$ & C & C${}^{p}$ & C & C${}^{p}$ & C & C${}^{p}$\\
  \hline\hline
  vpr & 4557&4557 & \cellcolor{good}165 & \cellcolor{good}0 & 275&275 & 298&297\\
  mesa & 13414&13420 & \cellcolor{good}939 & \cellcolor{good}0 & \cellcolor{good}1401& \cellcolor{good}1276 & \cellcolor{good}2265& \cellcolor{good}2133\\
  fuzz1 & 119&118 & \cellcolor{good}40 & \cellcolor{good}0 & 17&17 & \cellcolor{good}17& \cellcolor{good}15\\
  fuzz2 & 404&404 & \cellcolor{good}148 & \cellcolor{good}0 & 115&115 & \cellcolor{good}74& \cellcolor{good}65\\
  fuzz3 & 1515&1515 & \cellcolor{good}533 & \cellcolor{good}0 & 456&461 & \cellcolor{good}267& \cellcolor{good}226\\
\end{tabular}
\label{fig:eval:alloc_hard}
\end{table}
\end{frame}


\end{document}
