\documentclass[runningheads, orivec]{llncs}
%
%
\usepackage{geometry}
\geometry{
  a4paper,         % or letterpaper
  textwidth=15cm,  % llncs has 12.2cm
  textheight=24cm, % llncs has 19.3cm
  heightrounded,   % integer number of lines
  hratio=1:1,      % horizontally centered
  vratio=2:3,      % not vertically centered
}
\usepackage[T1]{fontenc}
% T1 fonts will be used to generate the final print and online PDFs,
% so please use T1 fonts in your manuscript whenever possible.
% Other font encondings may result in incorrect characters.
%
\usepackage{graphicx}
% Used for displaying a sample figure. If possible, figure files should
% be included in EPS format.
%
%%%%%%%%%%%%%%%%%%%%%%%%%%%%%%%%%%%%%%%%%%%%%%%%%%%%%%%%%%%%%%%
%%%%%%%%%%%%%%%%%%%%%%%%% MY PACKAGES %%%%%%%%%%%%%%%%%%%%%%%%%
%%%%%%%%%%%%%%%%%%%%%%%%%%%%%%%%%%%%%%%%%%%%%%%%%%%%%%%%%%%%%%%
\usepackage{xspace}
\usepackage{subcaption}
%\usepackage[gray]{xcolor}
\usepackage{xcolor}
\usepackage{tikz}
\usetikzlibrary{calc, arrows.meta, math, positioning, decorations.pathreplacing, patterns, shapes}
\usepackage[colorlinks=true, linkcolor=blue, urlcolor=blue, citecolor = blue]{hyperref}
\usepackage{amsmath}
\usepackage{amssymb}
\usepackage{semantic}
\usepackage{mathtools}
\usepackage{todonotes}
\usepackage[acronym]{glossaries}
\usepackage{bbding}
\usepackage{orcidlink}
\usepackage[capitalize]{cleveref}
\usepackage{xpatch}
\usepackage{eulervm}

\usepackage{pgfplots}
\usepackage{pgfplotstable}
\pgfplotsset{compat=1.16}
\usepgfplotslibrary{groupplots}

\definecolor{codebg}{HTML}{eeeeee}
\usepackage{minted}
\setminted{fontsize=\small, bgcolor=codebg}
\usemintedstyle{tango}

\usepackage{newunicodechar}
\newunicodechar{Π}{\ensuremath{\Pi}}
\newunicodechar{Φ}{\ensuremath{\Phi}}
\newunicodechar{∗}{\ensuremath{\ast}}
\newunicodechar{⌜}{\ensuremath{\ulcorner}}
\newunicodechar{⌝}{\ensuremath{\urcorner}}
\newunicodechar{∃}{\ensuremath{\boldsymbol\exists}}
\newunicodechar{κ}{\ensuremath{\kappa}}
\newunicodechar{σ}{\ensuremath{\sigma}}
\newunicodechar{↪}{\ensuremath{\hookrightarrow}}
\newunicodechar{≡}{\ensuremath{\equiv}}
\newunicodechar{□}{\ensuremath{\square}}


% If you use the hyperref package, please uncomment the following two lines
% to display URLs in blue roman font according to Springer's eBook style:
\usepackage{color}
\renewcommand\UrlFont{\color{blue}\rmfamily}
\urlstyle{rm}

\glsdisablehyper
\newacronym{cfg}{CFG}{context-free grammar}

\begin{document}

\title{Modular I/O Reasoning in DimSum}
%
%\titlerunning{Abbreviated paper title}
% If the paper title is too long for the running head, you can set
% an abbreviated paper title here
%
\author{Alexander Loitzl\Envelope\orcidlink{0009-0002-7417-2537}}%
\authorrunning{A. Loitzl}
% First names are abbreviated in the running head.
% If there are more than two authors, 'et al.' is used.
%

\institute{Institute of Science and Technology Austria, Klosterneuburg, Austria\\
\email{alexander.loitzl@ista.ac.at}}
%
\maketitle              % typeset the header of the contribution
%
%
% Abstract: an introduction conveying the general scientific background, research methodology and overall investigation status of the overarching scientific question in the pertinent scientific field

\begin{abstract}
  This report saves as an artifact capturing the outcome and insights of the authors rotation in Michael Sammler's PLV group at ISTA. We give a brief background and motivation of the project and conclude with a discussion of learning outcomes and insights and open problems.
\end{abstract}

\section{Background}
We briefly summarize the motivation, inspiration and state-of-the art before detailing project details.
\subsection{Modular I/O Reasoning}
The motivation for the project comes from works for the verifast prover\cite{} where I/O reasoning is built into Hoare-style reasoning, rather than handling this outside.
\subsection{DimSum}
DimSum is a framework for multi-language reasoning realized in the Rocq theorem prover. Program semantics are defined as LTS (modules) which can interact with each other by synchronizing on events like Process algebra.
This way DimSum allows arbitrary interaction between modules as long as they agree on the kind of events. Primitve combinators of modules are used to build complex modules which translate the events such that modules can be linked semantically, or wrapper modules which translate the events from one language to another.

The recent works of introducing Hoare-style reasoning to DimSum are in a sense orthogonal in the sense that we want to abstract as much as possible of this away in order to focus on reasoning about functional correctness of a single module. To combine module-local reasoning and the power of DimSum, the notion of a switch is introduced in order to change from one module to the other. At the moment it requires to know...

\section{Rotation Project}

The main objective of the rotation (as stated in the pre-rotation protocol) was to get familiar with DimSum and the tools used in its design. Mainly the SSreflect tactic language and Separation logic and the Iris Proof Mode. The means to achieve this goal was through a succession of various small verification tasks, first in the original proof style and later with the newly, partially developped modular hoare reasoning style.

The small examples served as stepping stones to get a more fundamental understanding of the framework and included simple functions and specifications.

\subsection{Main Challenge}

The main challenge chrystalized rather quickly and is the correctness proof of the following program in DimSum's Rec Language.

The main problem worked on during the rotation was the proof for the following program. The module on the left, is the linked module of the echo program with a getc spec.
The programm captures the main challenges of the project well, since through the recursion, one can not use any shortcuts. All components must be used in such a way that the ownership of resources has not changed.

\begin{figure}[h]
  \centering
  \hspace{.3em}
  \begin{subfigure}[b]{0.21\textwidth}
    \centering
    \begin{minipage}{\textwidth}
      \centering
      \inputminted[bgcolor=codebg]{c} {echo.rec}
    \end{minipage}
  \end{subfigure}
  %% \hspace{3em}
  \hfill
  \begin{subfigure}[b]{0.35\textwidth}
    \centering
    \begin{minipage}{\textwidth}
      \centering
      \inputminted[bgcolor=codebg]{coq} {getc.spec}
    \end{minipage}
  \end{subfigure}
  \hfill
  \begin{subfigure}[b]{0.4\textwidth}
    \centering
    \begin{minipage}{\textwidth}
      \centering
      \inputminted[bgcolor=codebg]{coq} {echo_getc.spec}
    \end{minipage}
  \end{subfigure}

  \caption{code of CompCert (left) and (right)}
  %\label{fig:design:example3}
\end{figure}


\subsection{Switching Notation}
% TODO: parts of this should probably
One other interesting aspect of the rotation is how to communicate the intuition correctly. The main focus of DimSum, being a framework for multi-language reasoning, was modeling the interaction of different components and giving good stuff for that.
The outcome of the ongoing discussions is a notation proposed by Michael Sammler, which was so far only used in order present the final results to the group, but has not been tested within the Rocq theorem prover.

\begin{figure}[h]
  \centering
  \hspace{.3em}
  \begin{subfigure}[b]{0.48\textwidth}
    \centering
    \begin{minipage}{\textwidth}
      \centering
      \inputminted[bgcolor=codebg]{coq} {getc_prepost_short.v}
    \end{minipage}
  \end{subfigure}
  %% \hspace{3em}
  \hfill
  \begin{subfigure}[b]{0.42\textwidth}
    \centering
    \begin{minipage}{\textwidth}
      \centering
      \inputminted[bgcolor=codebg]{coq} {echo_prepost.v}
    \end{minipage}
  \end{subfigure}
  \begin{subfigure}[b]{0.75\textwidth}
    \centering
    \begin{minipage}{\textwidth}
      \centering
      \inputminted[bgcolor=codebg]{coq} {sim_getc.v}
    \end{minipage}
  \end{subfigure}
  %
  \caption{code of CompCert (left) and (right)}
  %\label{fig:design:example3}
\end{figure}



\section{Discussion}

Below we discuss the outcome and future directions. We also talk about challenges since it is closely related to future work. The tool should be usuable and one of the goal is to find the right level to talk about.

\subsection{Outcome}

A simple lemma which allows straight forward reasoning about the \textcolor{red}{compound} operation `TCallRet'.
We have sucessfully used in in multiple proofs with a straightforward corollary of TWaitingRaw

Fix $\Pi$ upfront, and additional definitions and lemmas about the linking module were added.

\subsection{Challenges \& Future Work}

A top down approach was chosen...
It is hard to do say what is the correct way to do it. It could have been nice to develop an intuition for switches this way, but didn't happen.

How to communicate...

Multilanguage...

\bibliography{references}

\end{document}
